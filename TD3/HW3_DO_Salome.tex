\documentclass{article}

    \usepackage[utf8]{inputenc}
    \usepackage[T1]{fontenc}
    \usepackage{aeguill}
    % \usepackage[francais]{babel}
    \usepackage[a4paper]{geometry}
    \usepackage{array}
    \usepackage{amsfonts}
    \usepackage{amsmath} 
    \usepackage{amssymb}
    \usepackage{amsthm}
    \usepackage{xspace}
    \usepackage{dsfont}
    \usepackage{collcell}
    \usepackage{datatool}
    \usepackage{enumitem}
    \usepackage{xstring}
    \usepackage{booktabs}
    \usepackage{environ}
    \usepackage{bbm} 
    \usepackage{hyperref}
    \usepackage{graphicx}
    \usepackage{caption}
    \usepackage{stmaryrd}
    \usepackage[dvipsnames]{xcolor}
    \usepackage{tikz}
    \usetikzlibrary{trees}
    \usepackage{ulem}
    \usepackage{cancel}
    \usepackage{pgfplots}
    % \usepackage{minted}
    % \usemintedstyle{monokai}
    \usepackage{multicol}
    
    \pgfplotsset{compat=newest}
    \usetikzlibrary{automata} % LATEX and plain TEX
    \usetikzlibrary[automata] % ConTEXt
    \usetikzlibrary{arrows}
    \usetikzlibrary{automata,arrows,positioning,calc}
    \xdefinecolor{vertf}{named}{OliveGreen}
    \xdefinecolor{rougef}{named}{BrickRed}
    \xdefinecolor{bleuf}{named}{BlueViolet}
    \newcommand{\Var}{vecteur aléatoire réel\xspace}
    \newcommand{\var}{variable aléatoire réelle\xspace}
    \newcommand{\ssi}{si et seulement si\xspace}
    \newcommand{\cad}{c'est-à-dire\xspace}
    \newcommand{\fdr}{fonction de répartition \xspace}
    \newcommand{\pp}{\mathbb P}
    \newcommand{\un}{\mathbbm{1}}
    \newcommand{\esp}{\mathbb E}
    \newcommand{\vari}{\mathbb V}
    \newcommand{\cov}{\text{Cov}} 
    \newcommand{\gras}{\textbf}
    \newcommand{\itemb}{\item[$\bullet$]}
    \newcommand{\rouge}{\textcolor{red}}
    \newcommand{\bleu}{\textcolor{blue}}
    \newcommand{\rougef}{\textcolor{rougef}}
    \newcommand{\vertf}{\textcolor{vertf}}
    \newcommand{\bleuf}{\textcolor{bleuf}}
    \newcommand{\limn}{\underset{n\rightarrow +\infty}{\lim}}
    \newcommand{\flechn}{\underset{n\rightarrow +\infty}{\longrightarrow}}
    \newcommand{\RR}{\mathbb R}
    \newcommand{\Q}{\mathbb Q}
    \newcommand{\N}{\mathbb N}
    \newcommand{\Z}{\mathbb Z}
    \newcommand{\R}{\mathbb R}
    \newcommand{\D}{\mathbb D}
    \newcommand{\C}{\mathbb C}
    \newcommand{\Rn}{\mathbb R^n}
    \newcommand{\Rp}{\mathbb R^p}
    \newcommand{\Rq}{\mathbb R^q}
    \newcommand{\brn}{\mathcal B(\mathbb R^n)}
    \newcommand{\brp}{\mathcal B(\mathbb R^p)}
    \newcommand{\brq}{\mathcal B(\mathbb R^q)}
    \newcommand{\br}{\mathcal B(\mathbb R)}
    \newcommand{\brbarre}{\mathcal B(\overline{\mathbb R}}
    \newcommand{\pps}{P-presque-sûrement\xspace}
    \newcommand{\mespos}{\mathcal M^+(\mathcal B(\mathbb R^n),\mathcal B(\overline{\mathbb R}))}
    \newcommand{\cvps}{\xrightarrow[n\rightarrow\infty]{p.s.}}
    \newcommand{\cvld}{\xrightarrow[n\rightarrow\infty]{L^2}}
    \newcommand{\cvlp}{\xrightarrow[n\rightarrow\infty]{L^p}}
    \newcommand{\cvp}{\xrightarrow[n\rightarrow\infty]{\mathbb P}}
    \newcommand{\cvloi}{\xrightarrow[n\rightarrow\infty]{\mathcal L}}
    \newcommand{\definition}{\vspace{0.5cm}\begin{tcolorbox}[colback=bleuf!5!white,colframe=bleuf!75!black,title=Définition]}
    \newcommand{\propriete}{\vspace{0.5cm}\begin{tcolorbox}[colback=bleuf!5!white,colframe=bleuf!75!black,title=Propriété]}
    \newcommand{\proprietee}{\vspace{0.5cm}\begin{tcolorbox}[colback=red!5!white,colframe=red!75!black,title=Propriété]}
    \newcommand{\theoreme}{\vspace{0.5cm}\begin{tcolorbox}[colback=red!5!white,colframe=red!75!black,title=Théorème]}
    \newcommand{\lemme}{\vspace{0.5cm}\begin{tcolorbox}[colback=red!5!white,colframe=red!75!black,title=Lemme]}
    \newcommand{\proposition}{\vspace{0.5cm}\begin{tcolorbox}[colback=red!5!white,colframe=red!75!black,title=Proposition]}
    \newcommand{\fin}{\end{tcolorbox}\vspace{0.5cm}}
    \newcommand{\preuve}{\noindent\uline{Preuve :}\xspace}
    \newcommand{\remarque}{\noindent\uline{Remarque :}\xspace}
    \newcommand{\exemple}{\noindent\uline{Exemple :}\xspace}
    \newcommand{\rappel}{\noindent\uline{Rappel :}\xspace}
    \newcommand{\notation}{\noindent\uline{Notation :}\xspace}
    
    
    % transposition de tableaux
    
    
    \usepackage{booktabs,array}
    \def\Midrule{\midrule[\heavyrulewidth]}
    \newcount\rowc
    
    \makeatletter
    \def\ttabular{%
    \hbox\bgroup
    \let\\\cr
    \def\rulea{\ifnum\rowc=\@ne \hrule height 1.3pt \fi}
    \def\ruleb{
    \ifnum\rowc=1\hrule height 1.3pt \else
    \ifnum\rowc=6\hrule height \heavyrulewidth 
       \else \hrule height \lightrulewidth\fi\fi}
    \valign\bgroup
    \global\rowc\@ne
    \rulea
    \hbox to 10em{\strut \hfill##\hfill}%
    \ruleb
    &&%
    \global\advance\rowc\@ne
    \hbox to 10em{\strut\hfill##\hfill}%
    \ruleb
    \cr}
    \def\endttabular{%
    \crcr\egroup\egroup}
    
    % Couleurs
    
    \usepackage{tcolorbox}
    
    
    
    %équivalent tilde : \sim
    %très petit, grand \ll et \gg
    %Pour les formules aller sur codecogs
    %Si "missing delimiter" mettre un point derrière le \right (car il finit la ligne cf forum) ou alors : ne pas oublier que c'est \right\}
    %ANSI : mettre latin1`
    %Pour environnement align : mettre & devant le = pour bien aligner 
    %Mettre le tableofcontents avant pour avoir une table avant le document et après sinon
    %Attention : bien lancer deux fois lors d'une modification de la table des matières
    %Attention : il ne capte pas les environnements dans les tableaux : faire des environnements tabularx
    
    \hypersetup{colorlinks=true,linkcolor=black}
    
    \usepackage{fancyhdr}
    \pagestyle{fancy}
    \lfoot{S. DO }
    \rfoot{\thepage}
    \cfoot{ }
    \lhead{COMPUTATIONAL STATISTICS}
    \chead{ }
    \rhead{ TD3 }
    
    \renewcommand{\footrulewidth}{1pt}
    
    \newcommand{\ind}{\setlength\parindent{0.5cm}} 
    
\begin{document}
    
\thispagestyle{empty}

\title{Computational statistics \\ Homework 3}
\author{Salomé Do}
\maketitle

\section*{Exercise 7.22. : Travelling salesman problem with Metropolis-Hastings Algorithm}



The travelling salesman problem (TSP) can be formulated as follows : 
a salesmans has to visit $N \in \mathbb{N}$ customers, all living 
in different places  $\{ 1, ..., N \}$. The salesmans minds the planet, 
so he doesn't want to spend too much gas visiting his customers : he wants 
to drive as few miles as possible. Thus, he must choose the shortest 
route to visit his $N$ customers. We simulate
 \footnote{The code can be found 
 \href{https://github.com/sally14/ComputationalStats/blob/master/TD3/travelling_saleman.ipynb}{here}
  } this problem with $N=30$ places 
randomly determined.
\begin{center}    
    \begin{multicols}{2}
        \captionof{figure}{Places to visit in a simulation of the TSP}
        \includegraphics[scale=0.5]{places.png} \\ \\
        \captionof{figure}{A random route to visit the places (d=16.4)}
        \includegraphics[scale=0.5]{random_route.png}
    \end{multicols}
\end{center}

The problem can be expressed as follows, considering $d$ a distance on 
all the places $1, ..., N$, and the permutations $\sigma \in \mathcal{S}^{N}$ 
which define the order in which the travelling salesman sees his customers: 

\[\underset{\sigma \in \mathcal{S}^N}{\min}  \sum_{i} d(i, \sigma(i)) \]


The TSP problem can be tackled through a Metropolis Hastings algorithm.
 Let's define:  \[H(\sigma) = \sum_{i} d(i, \sigma(i))\] 
 $H$ is the objective function to minimize, with regard to a permutation $\sigma$. \\ \\

Following a Simulated-Annealing scheme, we want to :

At step $i$, given a permutation $\sigma_i$:
\begin{enumerate}
    \item  Simulate $\zeta$ a new permutation candidate, from an 
    instrumental density $g(|\zeta-\sigma_i|)$
    \item Compute: $$ \rho_i = min\{\exp (\frac{\Delta h_i}{T_i}) , 1\}$$ 
    and take $\sigma_{i+1} = \zeta$ with probability $\rho_i$, $\sigma_{i+1} = \sigma_{i}$ otherwise. 
    \item Update $T_i$ to $T_{i+1}$
\end{enumerate}


To implement this algorithm, two things are to be precised :
 the instrumental density $g$, and the 'temperature' function $T_t$.
 First, we represent a permutation $\sigma \in \mathcal{S}^N$ as a
 vector of dimension $N$, in the form of : 
$$\sigma = \left ( \begin{array}{c} \sigma(1) \\ \sigma(2) \\ \vdots \\ \sigma(N) \end{array} \right )$$ 

At each step $i$, we want to randomly generate a candidate
permutation $\zeta$, close to $\sigma_i$.
An idea for that is to randomly chose a pair of coordinates 
in the $\sigma_i$ vector and permute them. This randomly
generates a new permutation $\zeta$ close to $\sigma_i$.
Regarding the temperature function $T_t$, we can choose, as a first guess, 
the example from section $5.2.3$, namely $T_t = \frac{1}{log(t)}$. \\ \\
We then run the algorithm on our toy example generated in Figure 1, with $100000$ simulations.
\begin{center}   
    \begin{multicols}{2}
        \captionof{figure}{Route after 100 000 iterations}
        \includegraphics[scale=0.5]{end_route} \vspace{0.5cm}
        \captionof{figure}{Shortest route amongst all (d=4.9)}
        \includegraphics[scale=0.5]{min_route}
    \end{multicols}
\end{center}
The first thing to note is that the last route isn't the shortest route amongst
the simulations. Figure 5, which plots each simulation's distances, gives an 
idea of the convergence of the algorithm. In our case, the total distance
of the route is quickly reduced, but the convergence seems to be pretty poor. 

\begin{center}
    \captionof{figure}{Total distance for each route generated by the algorithm}
    \includegraphics[scale=0.7]{convergence.png}
\end{center}

As this might be due to the temperature function, we chose to explore other 
temperature functions. We first tried functions of the form $T_t = \frac{1}{\log(t^a)}, a \in \N$.
Although $a$ acts just as a regularization constant 
(since $\frac{1}{\log(t^a)} = \frac{1}{a\log(t)}$), the parameter changes the appearance
of the convergence plot. We tested value from $1$ to $100$ for $a$, and averaged the results of 
20 runs of the algorithm per $a$ value. The algorithm ran $1000$ Simulated-Annealing
iterations per simulation. The value giving the shortest route in 
average was $a=86$, and the shortest distance was $d=6.22$ (Figure 7) amongst 
20 runs for this $a$ value, thus a worse result than before. However, the convergence for 
$a=86$ (Figure 6) was clearer. 


\begin{center}
    \begin{multicols}{2}
        \captionof{figure}{Convergence for $a=86$}
        \includegraphics[scale=0.5]{convergence2} \vspace{0.5cm}
        \captionof{figure}{Best route for all 20 runs in $a=86$}
        \includegraphics[scale=0.5]{shortest_route}
    \end{multicols}
\end{center}

We gave another try to another form of temperature function : 
$T_t = \frac{1}{t^2}$. Running $100000$ simulations gives the following convergence
scheme (Figure 8), which is satisfying. The shortest route has a total 
distance of $5.33$ (Figure 9), which is near the results of the first temperature function 
$T_t = \frac{1}{\log(t)}$. The $T_t = \frac{1}{t^2}$ temperature function has thus the advantage 
of giving good results, while showing less variability than the previous functions.


\begin{center}
    \begin{multicols}{2}
        \captionof{figure}{Convergence scheme}
        \includegraphics[scale=0.5]{convergence3} \vspace{0.5cm}
        \captionof{figure}{Shortest route (d=5.33)}
        \includegraphics[scale=0.5]{shortest_route2}
    \end{multicols}
\end{center}

\section*{Exercise 10.10 Multi-stage Gibbs Sampler on Clinical Mastisis Data}

\subsection*{Overview of the problem}

Let $X_i, i=1, ..., m$ be the number of cases of mastisis (an inflammation 
caused by infection) in herd $i$. Assuming that the occurrence or not of a
 mastisis for a single animal is a Bernouilli random variable, a model for
  the herd could be:   $X_i \sim \mathcal{P}(\lambda_i)$,
where $\lambda_i$ is the infection rate. The main problem here is the
 independence which could lead to larger parameter estimates variances. 
 Schukken et al. 1991 use thus a hierarchical model :

\begin{align*}
    X_i &\sim \mathcal{P}(\lambda_i) \\
\lambda_i &\sim \mathcal{G}a(\alpha, \beta_i)  \\
\beta_i  &\sim \mathcal{G}a(a, b) \\
\end{align*}

\noindent The objective is to estimate $\lambda_i$ by simulation,
 using the Gibbs sampler.

\subsection*{Gibbs Sampler}

In the Multi-stage Gibbs Sampler context, we want to simulate
 $X = (X_1, ..., X_p)$ and we suppose that we are able to simulate
  from univariate conditionnal densities $f_1, ..., f_p$, so that :
$$X_i | x_1, x_2, ..., x_{i-1},x_{i+1}, ...,x_p \sim f_i(x_i | x_1, x_2, ..., x_{i-1},x_{i+1}, ...,x_p )$$
The Gibbs Sampler simulates $x^{t+1}$ from $x^{t}$ by doing the
 following updates at time $t$ : 

Generate : 
\begin{itemize}
    \item $X_1^{t+1} \sim f_1(x_1 | x_2^t, ..., x_p^t) $
    \item $X_2^{t+1} \sim f_2(x_2 | x_1^{t+1},x_3^t ..., x_p^t) $
    \item ...
    \item $X_p^{t+1} \sim f_p(x_p | x_1^{t+1}, ..., x_{p-1}^{t+1}) $
\end{itemize}

In our case, $\lambda_i, \beta_i$ are to be simulated. Let's first compute their posterior densities. 

\subsection*{(a) Posterior densities computation}

We know that $\lambda_i \sim \mathcal{G}a(\alpha, \beta_i) $. Thus, 
$$\pi(\lambda_i = \lambda | \alpha, \beta_i) = \lambda^{\alpha-1}\frac{\beta_i^{\alpha}e^{-\beta_i \lambda}}{\Gamma(\alpha)}, \forall \lambda \in \mathbb{R}^+$$
Additionally : $f(x_i|\lambda_i = \lambda, \alpha, \beta_i) = \frac{\lambda^{x_i}}{x_i!}e^{-\lambda}$. 
Then, from Bayes Theorem :
$$\pi(\lambda_i = \lambda | x , \alpha, \beta_i) = 
\frac{
    f(x_i| \lambda_i = \lambda, \alpha, \beta_i)
     \pi(\lambda_i=\lambda | \alpha, \beta_i)}
{\underset{A}{\underbrace{
    \int_0^{\infty}
    f(x_i|  \lambda_i = s, \alpha, \beta_i)
    \pi(\lambda_i = s| \alpha, \beta_i) 
ds}}}$$.
Let's first compute the integral $A$. 
\begin{align*}
    A &= \int_0^{\infty}
            \frac{
                s^{x_i+\alpha-1} \times e^{-(\beta_i+1)s} \times \beta_i^{\alpha}
            }
            {x_i! \times \Gamma(\alpha)}
            ds \\
      &= \frac{\beta_i^{\alpha}}{ x_i! \times \Gamma(\alpha) }
         \int_0^{\infty}
            s^{x_i+\alpha-1} \times e^{-(\beta_i+1)s}  ds \\
     &= \frac{\beta_i^{\alpha}}{ x_i! \times \Gamma(\alpha) \times \beta_i^{x_i+\alpha}}
        \int_0^{\infty} u^{x_i+\alpha-1} e^{-u} du \ \ \ \ \ \ \ \ \ \ \ \leftarrow u:= (\beta_i+1)s \\
    &=  \frac{\beta_i^{\alpha}}{ x_i! \times \Gamma(\alpha) \times \beta_i^{x_i+\alpha}}
        \times \Gamma(x_i+\alpha) \\
\end{align*}

\noindent Then, 

\begin{align*}
    \pi(\lambda_i = \lambda | x , \alpha, \beta_i) &=
    \frac{ x_i! \times \Gamma(\alpha) \times \beta_i^{x_i+\alpha}}
    {\beta_i^{\alpha}\times \Gamma(x_i+\alpha)}
    \lambda^{\alpha-1}\frac{\beta_i^{\alpha}e^{-\beta_i \lambda}}{\Gamma(\alpha)}
     \frac{\lambda^{x_i}}{x_i!}e^{-\lambda}\\
     &= \frac{\lambda^{x_i+\alpha-1} (\beta_i+1)^{x_i+\alpha} e^{-(\beta_i+1)\lambda}}
     {\Gamma(x_i+\alpha)}
\end{align*}
Which is the density of a $\mathcal{G}a(x_i+\alpha, \beta_i+1)$.

\noindent Now, show that $\pi(\beta_i|x, \alpha, a, b, \lambda_i) =
 \mathcal{G}a(\alpha + a, \lambda_i + b)$. Denote, for clarity:  \[f(\beta) = 
 \pi(x_i | \lambda_i, \beta_i= \beta, \alpha, a, b )
        \pi(\lambda_i|\beta_i=\beta, \alpha, a, b)
        \pi(\beta_i = \beta| a, b), \forall \beta \in \R^+\]
         From Bayes Theorem, for any $\beta \in \R^+$,  
 \begin{align*}
    \pi(\beta_i = \beta | x, \alpha, a, b, \lambda_i) &=
        \frac{f(\beta)}
        {\int_0^\infty f(s)ds}  
 \end{align*}
 Using the respective laws of $X_i, \lambda_i, \beta_i$ : 
 \begin{align*}
     f(\beta) &= \frac{\lambda_i^{x_i}}{x_i!} 
                  e^{-\lambda_i}
                  \lambda_i^{\alpha-1} 
                  \frac{\beta^\alpha e^{-\beta\lambda_i} }{\Gamma(\alpha)}
                  \beta^{a-1} 
                  \frac{b^a e^{-b\beta}}{\Gamma(a)} \\
             &= \underset{C}{\underbrace{\frac{\lambda_i^{x_i+\alpha-1}e^{-\lambda_i}b^a   }
             {x_i! \Gamma(\alpha) \Gamma(a)} } }
             \beta^{\alpha+a-1}
              e^{-\beta(\lambda_i+b)}
 \end{align*}
 With $C$ a constant regarding $\beta$. Thus, the integral value can be calculated as follows :
 \begin{align*}
    \int_0^\infty f(s)ds &= \int_0^\infty C 
                            s^{\alpha+a-1}
                            e^{-s(\lambda_i+b)} ds  \\
                        &= C \int_0^\infty \frac{1}{(\lambda_i+b)^{\alpha+a} }
                            t^{(\alpha+a)-1}e^{-t} dt \ \ \ \ \ \leftarrow t = s(\lambda_i+b) \\
                        &= \frac{C}{(\lambda_i+b)^{\alpha+a} } \Gamma(\alpha +a)
 \end{align*}
 Thus, 
 \begin{align*}
     \pi(\beta_i = \beta | x, \alpha, a, b, \lambda_i) &= 
       \frac{
            C \beta^{\alpha+a-1}
            e^{-\beta(\lambda_i+b)}
            (\lambda_i+b)^{\alpha+a} 
       }
       {C \Gamma(\alpha+a)} \\
    &=  \beta^{\alpha+a-1}
    \frac{
        (\lambda_i+b)^{\alpha+a} 
        e^{-\beta(\lambda_i+b)}
   }
   { \Gamma(\alpha+a)}
 \end{align*}
Which is the density of a $\mathcal{G}a(\alpha+a, \lambda_i+b)$. 

\subsection*{(b), (c) Gibbs sampler implementation}

We implemented \footnote{Code available \href{https://github.com/sally14/ComputationalStats/blob/master/TD3/gibbs_mastitis.ipynb}{here}} 
the Gibbs Sampler for $\lambda_i, \beta_i$ for $3$ herds (5, 15, 100), with recommended parameters
and $n=5000$ simulations. 
Results are plotted in Figure 10, Figure 11, Figure 12. 

Results show that the convergence for $\lambda, \beta$
is precise up to a $10^{-1} , 10^{-2}$ factor. 

\begin{center}
    \captionof{figure}{Herd 5}
    \includegraphics[scale=0.4]{herd5}
    \vspace{2cm}
    \captionof{figure}{Herd 15}
    \includegraphics[scale=0.4]{herd15}
    \captionof{figure}{Herd 100}
    \includegraphics[scale=0.4]{herd100}
\end{center}

\newpage
\subsection*{(d) Sensitivity to the parameters}
In this section, we explore the influence of the parameters 
on $\lambda_{100}, \beta_{100}$. We only play with one parameter at a time, 
and the results are given in Figure 13 and Figure 14.
\begin{center}
    \begin{multicols}{2}
        \captionof{figure}{Sensitivity of $\lambda_{100}$}
        \includegraphics[scale=0.5]{sens_lambda}
        \vspace{0.5cm}
        \captionof{figure}{Sensitivity of $\beta_{100}$}
        \includegraphics[scale=0.5]{sens_beta}
    \end{multicols}
\end{center}

First, we can remark $\lambda_{100}$ is only sensible to $a$ value,
 which might be unexpected, showing a real importance of $\beta_{100}$ 
 parameters
 for $\lambda_{100}$ in the hierarchical model. Second, $\beta_{100}$ 
 is very sensible to $\alpha$, and a little to extreme values of $a$.

This settings seems to indicate that $\alpha, a$ are important parameters, 
and that the Gibbs-sampler setting gives a strong importance to $\beta_i$ parameters
for $\lambda_i$, and vice-versa.

\section*{Exercise 9.4 Gibbs-Sampler - Metropolis Hastings relation}
In this exercise, we consider that we have a bivariate Gibbs Sampler, with 
$X \sim f(x|y), Y\sim f(y|x)$. 

\subsection*{(a) Kernel}
We want to show that $K(x, x') = g(x|x') = \int f(x'|y) f(y|x) dy$.
As the two-stages Gibbs Sampler consists of simulating, at each time step $t$, 
$X \sim f_{Y|X}(\dot|x_{t-1}, Y \sim f_{X|Y}(\dot|y_{t}$, the sequences
$(X_t, Y_t), (X_t), (Y_t)$ are markov chains.
Moreover, in the Metropolis-Hastings setup, we draw $x' \sim K(x, x') = g(x|x')$
with $K(x, x')$ the transition kernel of $(X_t)$. Thus, using the law of total probability,
for $x, x'$ : 
\begin{align*}
    K(x, x') &= f(x'|x) \\
            &= \frac{f(x, x')}{f(x)} \\
            &= \int \frac{f(x, x'|y)f(y)}{f(x)} dy\\
            &= \int f(x'|y)\frac{f(x|y)f(y)}{f(x)} dy \\
            &= \int f(x'|y) f(y|x) dy \\
\end{align*}
\subsection*{(b) $\rho$}

The Metropolis Hastings algorithm generates $X^{(t+1)} \sim g(x^{t+1}|x^t)$, and accepts 
$X=x^{t+1}$ with probability $\rho = \min(\frac{f(x^{t+1})/g(x^{t+1}|x^t)}
{f(x^{t})/g(x^{t}|x^{t+1})}, 1)$, and keep $X = x^t$ with probability $1-\rho$. 
The stationnary distribution $f(.) $ is also the marginal of $X$, as: 
\begin{align*}
    \int f(x)g(x'|x) dx &= \int f(x)\int  f(x'|y) f(y|x) dy  dx \ \ \text{ (using question (a))}\\
    &= \int f(y) f(x'|y) dy \\
    &= f(x')
\end{align*}
% \subsection*{(c) Acceptation of the Metropolis Hastings proposal}
% We first want to show that $f(x') / g(x'|x) = f(x)/g(x| x')$, i.e.  
% $f(x') g(x| x')= \int f(y)f(x'|y) dy$.
% We have : 
% \begin{align*}
%     f(x')g(x| x') &= 
% \end{align*}
\end{document}